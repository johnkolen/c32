\documentclass{article}
\usepackage{amsmath}
\usepackage{amsthm}
\usepackage{multicol}
\usepackage{graphicx}
\usepackage{float}

\newtheorem{theorem}{Theorem}
\newtheorem{lemma}{Lemma}
\newtheorem{corollary}{Corollary}

\title{A Space-Limited Algorithm for Computing Collatz Sequences}
\author{John F. Kolen\\
  New Mexico Institute of Mininig and Technolgy \\
  Socorro, NM
  }

\date{\today}
\begin{document}

\maketitle


\begin{abstract}
  This paper describes a space-bounded algorithm for computing Collatz
  sequences.  It relies on performing calculations using a double-base number
  system, specifically ${2,3}$-integers. This
  representational scheme simplifies the Collatz operations on $x$ of $3x+1$
  and $x/2$ to left-shift plus 1 and down-shift. These operations do not
  destroy terms and will consume unbounded space if left unchecked. It is
  possible to create a trapezoidal boundary and fold terms back onto this constrained
  set of coefficient bits. For any positive integer, $x$, it is possible to perform Collatz
  iterations within a $n+2$ by $n$ trapezoid area where $n$ is the size of the binary
  representation of $x$. It will be shown that any Collatz sequence starting from a positive
  integer will eventually cycle. Additionally, bounds for both sequence maximum value and length will
  be provided.
\end{abstract}

\section{Introduction}

The Collatz conjecture is known by many names: the Ulam problem, the Syracuse
problem, and the $3n+1$ conjecture\cite{lagarias2023ultimate}. It is an unsolved problem in mathematics
that has been around for over eighty years regarding sequences of number that
follow a simple set of rules.
\begin{equation}
  C(x) = \begin{cases}
    3 x + 1 & \text{if } x \text{ is odd}\\
    x / 2 & \text{otherwise}
    \end{cases}
\end{equation}
The conjecture is that for positive integers, iterating this function will eventually
result in a cycle. Decades of simulations have shown that the Collatz path from any
starting point eventually enters the cycle $4\rightarrow2\rightarrow1\rightarrow4$.

In this paper, a similar version will be used that recognizes that every $3x+1$
operation is always followed by a division since the value is even ($3(2x+1) + 1=6x + 4$).
Known as the Syracuse variation, the function is rewritten to account for eliding the
two operations.

\begin{equation}
  C(x) = \begin{cases}
    (3 x + 1) / 2 & \text{if } x \text{ is odd}\\
    x / 2 & \text{otherwise}
    \end{cases}
\end{equation}

This variation also experiences eventually cyclic paths of the form
$2\rightarrow1\rightarrow2$.  It was also used by Terrance Tao in 2019 to
produce the most significant result on this topic came from when he proved that
almost all starting numbers will come close to one \cite{tao2022}.

This paper describes a space-bounded algorithm for computing the Collatz
sequences.  By utilizing a double-base number system to carry out calculations,
it is possible to construct a trapezoidal-boundary that limit the bits used to
represent sequence elements.  The constraints are determined \emph{a priori}
from the length of the binary representation of the starting number. The two
transformations of the Collatz function become local shift operations in the
assumed number system. The real complexity of the technique is the return of
any escaping bits back into the fixed region.  After recognizing that the
trapezoidal region is closed to the Collatz function, it is clear that Collatz
sequences must eventually cycle.

\section{Representing Integers}

A double-base number system uses combinations of powers of two relatively prime
numbers\cite{dimitrov2017}.
The formulation of ${3,2}$-integers is captured by the summation
\begin{equation}
  \sum_{i=0}^n \sum_{j=0}^n x_{ij} 2^i3^j
\end{equation}

where $x_{ij}$ are either zero or one and will be referred to as ``bits''. It
will be useful to separate an integer value from it's bit-wise
representation. A Cartesian-layout will be used for visualizations (Table~\ref{tbl:pvs}).
Double-base number systems provide multiple ways of
representing an integer value, unlike single base systems. For instance, the
number $4$, a power of two, can be represented as $1+3$, the sum of two powers
of three. While there is only one way to represent 127 in binary, there are
several ways to capture 127 as a ${3,2}$-integer (Figure~\ref{fig:127}). Double-base
notation also allows for fractional components when the place value powers dip below
zero. For instance, $5$ can be represented by $2^-13^0+2^-13^2$.

\begin{table}
  \begin{center}
  \begin{tabular}{lccccc}
  4 & $2^43^0$ & $2^43^1$& $2^43^2$& $2^43^3$& $2^43^4$ \\
  3 & $2^33^0$ & $2^33^1$& $2^33^2$& $2^33^3$& $2^33^4$ \\
  2 & $2^23^0$ & $2^23^1$& $2^23^2$& $2^23^3$& $2^23^4$ \\
  1 & $2^13^0$ & $2^13^1$& $2^13^2$& $2^13^3$& $2^13^4$ \\
  0 & $2^03^0$ & $2^03^1$& $2^03^2$& $2^03^3$& $2^03^4$ \\
  $i/j$ & 0 & 1 & 2 & 3 & 4
  \end{tabular}
  \end{center}
  \caption{
    Bit coefficients for the planar representation of ${3,2}$-integers.}
  \label{tbl:pvs}
\end{table}

\begin{figure}
  \begin{multicols}{3}
\begin{verbatim}
1
1
1
1
1
1
1
\end{verbatim}
Base 2
\columnbreak
\begin{verbatim}
1
0
0
0
001
0
0001
\end{verbatim}
Minimal number of bits
\columnbreak
\begin{verbatim}





001
10011
\end{verbatim}
Base 3
  \end{multicols}

  \caption{Several ways to represent 127 as ${3,2}$-integers.}
  \label{fig:127}
\end{figure}

Under the adopted ${3,2}$ notation, multiplication and division by either $2$
or $3$ is a shift. Specifically, $3x$ is a right-shift ($x\rightarrow$) and
$x/2$ is a down-shift ($x\downarrow$) under the Cartesian-layout of bits. In
the binary system, division is still a shift, but multiplication by three
requires a shift and add ($3x=2x+x$). Now, the Collatz transformation becomes

\begin{equation}
  C(x) = \begin{cases}
    (x\rightarrow + 1)\downarrow & \text{if } x \text{ is odd}\\
    x\downarrow & \text{otherwise}
    \end{cases}
\end{equation}
It is possible to introduce a new operator, $\searrow$, that captures both the left and
down shifts. This simplification yields the final form of the Collatz function for this paper.
\begin{equation}
  C(x) = \begin{cases}
    x\searrow + 1/2 & \text{if } x \text{ is odd}\\
    x\downarrow & \text{otherwise}
  \end{cases}
  \label{eqn:c}
\end{equation}
In this formulation, all processing after the conditional is local. Broadcasting the
operation selection triggers bit shifting across the representation. There are no carries
or borrows that a single-base system would have to worry about.

Determining parity of ${3,2}$-integers is a bit more involved. The only terms that impact
parity are those of the form $x_{0j}3^j$. It is necessary to count these terms modulo two
in order to compute parity. It will be useful in the following discussion to consider
column and row values. A column value is the sum of active place values for a fixed power of
three, $3^j\sum_{i=0}^{n}2^i$, where $j$ is the column number. A row value corresponds to
sum of place values for a fixed power of two.

It should be clear that under this processing scheme, the Collatz function
never destroys a bit once it has been added to the tableau. By allowing the
notation to venture into fractional values, bit patterns propagate unchanged
across the representational plane as the Collatz function iterates.  At any
given time, the number of bits is the number of starting bits and the number of
multiplies that have happened so far. The final $2$-$1$ cycling of the computation
system seen in Figure~\ref{fig:21} is one of value and not representation.

\begin{figure}
  \begin{multicols}{5}
\emph{2}
\begin{verbatim}
1
0
----



\end{verbatim}
\columnbreak
\emph{1}
\begin{verbatim}
0
1
----



\end{verbatim}
\columnbreak
\emph{2}
\begin{verbatim}
0
0
----
11


\end{verbatim}
\columnbreak
\emph{1}
\begin{verbatim}
0
0
----
0
11

\end{verbatim}
\columnbreak
\emph{2}
\begin{verbatim}
0
0
----
1
0
011
\end{verbatim}
\columnbreak
  \end{multicols}
  \caption{A $2-1$ oscillation. The dashes separate the integer part above
    from the ``fractional'' part below. Sometimes integers can be represented
    as sums of fractional terms. The middle $2$ is represented by $1/2 + 3/2$.}
  \label{fig:21}
\end{figure}

\section{Enforcing Boundaries}

Implementing the Collatz process using ${3,2}$-integers produces
unbounded propagation of bit patterns as sequences enter the observed cycle.
Only the exponential decreasing place value keeps the overall value
along its expected path. Recall that proving the conjecture will involve
cyclic Collatz sequences. These cycling values will not appear as repetition
of bit structures as a consequence of non-uniqueness of double-based integers.
It is imperative to constrain these wandering bits to a fixed playground. This
section proposes such a boundary and techniques for keeping bits within it without
affecting the overall value.

Let $x$ be the starting value of a Collatz sequence and $n$ be the number of
bits in the binary representation of $x$ ($\lceil\log_2x\rceil$). Consider the
trapezoid of bits illustrated in Figure~\ref{fig:boundary}. This $n+2$ by $n$
trapezoid, containing $n(n+5)/2$ bits, is capable of containing applications of
$C(x)$ on the value of the current representation.  The term
${3,2}$-trapezoid will refer to a ${3,2}$-integer representation with these contraints.
In order to justify this claim, the effects of the two operations must be examined.


\begin{figure}[H]
  \begin{center}
    \includegraphics[scale=0.5]{"Images/boundary"}
  \end{center}
  \caption{The Collatz trapezoid for ${3,2}$-integers. The origin ($2^03^0$) is
    the lower left corner.}
  \label{fig:boundary}
\end{figure}

The division step is a downward shift of the bits in the
representation. Figure~\ref{fig:divide} shows the effect of this shift on
the trapezoid. While most of the bits remain within the original area, a row of
bits escape into the fractional area. Notice that the top-most bit of each column is
vacated during the shift, as well. Let's compare the value of the escapee bits with that of
the vacated bits.

\begin{align*}
  \sum_{j=0}^{n-1}2^{-1}3^j &< \sum_{j=0}^{n-1}2^{n+1-j}3^j \\
  \frac{3^n-1}{2} &< 4(3^n - 2^n) \\
  4\cdot2^n - 1 &< 7\cdot3^n
\end{align*}

This inequality holds for all positive $n$ and shows that it is possible to add
the value of the escapee row back into the core representation and keep them
within the boundary. The escapees can be returned to the trapezoid by finding a
non-fractional representation of the escape row value and adding to the
trapezoid. This addition will be column-wise. That is carries will remain within
the column and percolate bits upward. The escapee representation should
use column values less than $2^{n+2-j}$, where $j$ is the column number.
A simple technique for this process
is to take the left most bits whose value is less than $2^{n+1}$, convert it to
binary, and add it to the first column value. The resulting column will be less
than the boundary restriction of $2^{n+2}$. This effectively rotates the base-$3$
representation of the row into a binary representation of the column.
The same rotation can be performed
with the remaining escape bits: convert the value to binary and add it to the
appropriate column.  It is possible that rotations will leave two bits in the
fractional area.\footnote{Note that the escapee row will contain an
even number of bits. This row was previously the first row and was used to determine
the parity of the value.  Partitioning this row could produce two collections of
an odd number of bits.} To return these bits it is necessary to expand $(1+3^k)/2$
into $1 + \sum_{j=0}^{k-1}3^j$, effectively adding one to columns $1$ through
$k-1$ and two to the zero column.  Divisions, therefore, can be safely performed
on trapezoidally-constrained representations.

\begin{figure}[H]
  \begin{center}
    \includegraphics[scale=0.5]{"Images/division_bold"}
  \end{center}
  \caption{The downward shift of the $x/2$ operation. The bottom row ($\sum_{j=0}^{n-1}3^j$)
    shifts downward as $\sum_{j=0}^{n-1}2^{-1}3^j$).}
  \label{fig:divide}
\end{figure}

With division out of the way, the odd-value operation can now be addressed. The combined
multiplication and division step is a diagonal shift with ${3,2}$-integers. The
``plus one'' becomes the addition of one-half in this representational
space, as seen in Equation~\ref{eqn:c}. Figure~\ref{fig:multiply} captures the diagonal shift and addition that
occurs to an odd-value (an odd number of bits on the first row) during this
step. At this point, the escapees include both bits on the right-side of
the trapezoid and the previously seen fractional row.  Note that the zero column is vacated as well as the diagonal. As before,
comparison of the escapees and the vacancies
\begin{align*}
  \sum_{j=0}^{n}2^{-1}3^j + 3^n + 2\cdot3^n &< \sum_{i=0}^{n}2^i + \sum_{j=0}^{n-1}2^{n+1-j}3^j \\
  \frac{3^{n+1}-1}{4} + 3^{n+1} &< 2^{n+1} - 1 + 4(3^n - 2^n) \\
  3^{n+1}-1 + 4\cdot3^{n+1} &< 4\cdot2^{n+1} - 4 + 16\cdot3^n - 16\cdot2^n \\
  8\cdot2^n + 3  &< 3^n  \\
\end{align*}

This inequality holds for $n \ge 3$ and demonstrates that there is enough room
to return the escapee bits back to the trapezoid. One process for this return
would be find a subset of bits in the vacant area and use it to represent the value
of the escapees. If no such subset exists, then a sequence of column values limited
by $2^{n+2-j}-1-col(j)$, where $col{j}$ is the value of the $j$th column after the shift
not including the fractional area will suffice. Thus, the multiplication escapees can be returned to
the trapezoidal playground and supports the following lemma.
\begin{lemma}
  A $n+2$ by $n$ trapezoid of ${3,2}$-integer bits is closed under the Collatz function, $C(x)$, when $n\ge3$.
  \label{lem:closure}
\end{lemma}


\begin{figure}[H]
  \begin{center}
    \includegraphics[scale=0.5]{"Images/multiply_bold"}
  \end{center}
  \label{fig:multiply}
\end{figure}

\section{Cycling Sequences}

The previous sections laid the foundation for proving the following theorem.

\begin{theorem}
  {For any positive integer $x$, the Collatz sequence starting at $x$ will eventually
  cycle.}
\end{theorem}

\begin{proof}
Given a positive integer $x$, construct a $n+2$ by $n$ ${3,2}$-trapezoid where
$n=max(3, \lceil \log_2x \rceil)$.  Initialize the bits so that the
structure's value is $x$ by filling the zero column with the binary expansion
of $x$. Since this trapezoid is closed under the Collatz transformations by
Lemma~\ref{lem:closure}, there are a finite number of possible states the
iteration could produce. Eventually, a state will repeat and the corresponding
Collatz sequence will reach a fixed point or cycle. The repeated state is not
fixed point as the two fixed points of $C(x)$ are $-1$ and $0$, neither of
which are a positive integer. Therefore, the sequence starting with $x$ eventuall cycles.
\end{proof}

\begin{corollary}
  If a positive integer $x$ can be represented in a $n-2$ by $n$ ${3,2}$-trapezoid,
  \begin{itemize}
  \item the maximum sequence length is less than $\sqrt{2^{n(n+5)}}$ (possible states),
  \item the maximum value of the sequence is bound by $3(3^{n+2}-1)/2 - 2^{n+2}-1-2\cdot3^{n+1}$
    (maximum trapezoid value).
  \end{itemize}
\end{corollary}

\begin{corollary}
\end{corollary}

\begin{corollary}
\end{corollary}

\section{Conclusion}

The difficulty of this long-standing problem was not a lack of mathematics, but
one of representation. The key to this proof was selecting a numbers system where
the transformations become trivial. Using binary makes division easy and base-3 simplifies
multiplication. The secondary operation in either requires nonlocal carries and borrows.
The method described above shifted the complexity of the iteration to the normalization
process. Hopefully, this technique can be applied to similar questions of cyclic iterations.

\bibliographystyle{plain}
\bibliography{collatz}

\end{document}
